% !TEX encoding = UTF-8 Unicode
%\documentclass{report}
\documentclass{memoir}

\usepackage[utf8]{inputenc}
\usepackage[T2A]{fontenc}
\usepackage{multicol}

%\usepackage[letterpaper, total={6in, 8in}]{geometry}
\usepackage[letterpaper, left=5mm, top=5mm, bottom=5mm, right=5mm ]{geometry}

% use to assign specific font. need xelatex processor
\usepackage{fontspec}

\setromanfont{Times New Roman}
\setsansfont{Arial}
\setmonofont[Color={0019D4}]{Courier New}

% set sans family as default (arial)
\renewcommand{\familydefault}{\sfdefault}


\title{Songbook v3}


\begin{document}
\tableofcontents
\clearpage
\chapter*{ВЫСОЦКИЙ}
\clearpage

\section*{Бег иноходца}

\begin{multicols}{2}
\noindent
{\huge
Я скачу, но я скачу иначе\\
По камням, по лужам, по росе.\\
Бег мой назван иноходью, значит\\
По-другому, то есть - не как все.\\

\noindent
Мне набили раны на спине,\\*
Я дрожу боками у воды.\\
Я согласен бегать в табуне —\\
Но не под седлом и без узды!\\

\noindent
Мне сегодня предстоит бороться \\
Скачки! Я сегодня фаворит.\\
Знаю, ставят все на иноходца,\\
Но не я -жокей на мне хрипит!\\

\noindent
Он вонзает шпоры в ребра мне,\\
Зубоскалят первые ряды…\\
Я согласен бегать в табуне —\\
Но не под седлом и без узды!
 }
\columnbreak  
    
Нет, не будут золотыми горы 
Я последним цель пересеку:
Я ему припомню эти шпоры,
Засбою, отстану на скаку!..
Колокол! Жокей мой на коне,
Он смеётся в предвкушеньи мзды.
Ох, как я бы бегал в табуне 
Но не под седлом и без узды!
Что со мной, что делаю, как смею!
Потакаю своему врагу!
Я собою просто не владею —
Я прийти не первым не могу!
Что же делать? Остаётся мне
Вышвырнуть жокея моего
И бежать, как будто в табуне, —
Под седлом, в узде, но без него!
Я пришёл, а он в хвосте плетётся
По камням, по лужам, по росе…
Я впервые не был иноходцем —
Я стремился выиграть, как все!
Я впервые не был иноходцем —
Я стремился выиграть, как все!
\end{multicols}

\end{document}
